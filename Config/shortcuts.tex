%! TEX root = main.tex
% !TeX spellcheck = en_US
%%%%%%%List of useful commands and definition of math operators%%%%%%%

% Redefining \caption command to include the short title in bold in the caption
\newcommand{\mycaption}[2][]{\caption[#1]{\textbf{#1} #2}}

% Useful shortcuts, for repeated terms
% for those that are acronyms
\newcommand{\tmd}{\gls{tmd}}
\newcommand{\dft}{\gls{dft}}
\newcommand{\mbpt}{\gls{mbpt}}
\newcommand{\rpa}{\gls{rpa}}
\newcommand{\bz}{\gls{bz}}
\newcommand{\lcao}{\gls{lcao}}
\newcommand{\rhs}{\gls{rhs}}
\newcommand{\lhs}{\gls{lhs}}
\newcommand{\hf}{\gls{hf}}
\newcommand{\hbn}{\gls{hbn}}
\newcommand{\bse}{\gls{bse}}
\newcommand{\gs}{\gls{gs}}
\newcommand{\soc}{\gls{soc}}

% those that are commonly used
\newcommand{\abinit}{\textit{ab-initio}}
\newcommand{\cti}{\textbf{continuous} translation invariance}
\newcommand{\dti}{\textbf{discrete} translation invariance}
\newcommand{\qp}{quasi-particle}
\newcommand{\wf}{wave-function}

% Defining units to be used.
\DeclareSIUnit\angstrom{\text{Å}} % without it compiler giver warning of "\r command not defined" whenever \AA is used
\DeclareSIUnit{\angstrominverse}{\angstrom\tothe{-1}}

% Macro to attribute units to vectors
\newcommand{\SIvector}[2]{\ensuremath{(#1)~\SI{}{#2}}}

%Mathematical functions/operators
\DeclareMathOperator{\arccosh}{arcCosh}
\DeclareMathOperator{\arcsinh}{arcsinh}
\DeclareMathOperator{\arctanh}{arctanh}
\DeclareMathOperator{\arcsech}{arcsech}
\DeclareMathOperator{\arccsch}{arccsch}
\DeclareMathOperator{\arccoth}{arcCoth}

%Physical Operators

%Hamiltonian
\newcommand{\Ha}{\hat{H}}
%
%%Single-particle Hamiltonian
\newcommand{\Ho}{\hat{H}_0}
%
%%Interaction term
\newcommand{\V}{\hat{V}}

%New emphasizing box
\newtcbox{\mymath}[1][]{%
    nobeforeafter, math upper, tcbox raise base,
    enhanced, colframe=blue!30!black,
    colback=blue!30, boxrule=1pt,
    #1}


%Command just for the GW approximation, either italic or not
%\newcommand{\GW}{\textit{GW}}
\newcommand{\GW}{$GW$}

%Defines command for vector fields/vectors that are variables (f.eg.s electric field, magnetic field, etc...)
%\newcommand{\i}[1][]{\bm{#1}}

%null vector
\newcommand{\bzero}{\vb{0}}

%tensor
\newcommand*{\rttensor}[1]{\overline{\overline{#1}}}

%Direct Bravais lattice vector
\newcommand{\bR}{\vb{R}}

%Exciton momentum
\newcommand{\bQ}{\vb{Q}}

%Vector of the motif of the unit cell/Vector of the basis within the unit cell
\newcommand{\bt}{\vb{t}}

%reciprocal lattice vectors
\newcommand{\bb}{\vb{b}}

%Vectors of the reciprocal lattice (basis)
\newcommand{\bG}{\vb{G}}

%normal vector -> n in bold
\newcommand{\bn}{\vb{n}}

%position vector -> r,x,y in bold
\newcommand{\br}{\vb{r}}

\newcommand{\bx}{\vb{x}}

\newcommand{\by}{\vb{y}}


%velocity vector -> v in bold
\newcommand{\bv}{\vb{v}}

%Electric field vector -> E in bold, italic or not
%\newcommand{\bE}{\vb{E}}
%\newcommand{\bE}{\bm{E}}

% Conditional definition of vector commands
\newif\ifusebm
\usebmtrue % Set it to true (\usebmfalse for false)


\ifusebm
    %Electric field vector
    \providecommand{\bE}{\bm{E}}

    %Potential vector
    \providecommand{\bA}{\bm{A}}
    \providecommand{\hA}{\hat{\bm{A}}}
    \providecommand{\nhA}{\bm{A}} %In case I want to remove the hat and recover it back again

    %Magnetic induction field vector
    \providecommand{\bB}{\bm{B}}


    %Electric displacement vector
    \providecommand{\bD}{\bm{D}}

    %Potential vector
    \providecommand{\bA}{\bm{A}}
    \providecommand{\hA}{\hat{\bm{A}}}
    \providecommand{\nhA}{\bm{A}} %In case I want to remove the hat and recover it back again

    %Magnetic induction field vector
    \providecommand{\bB}{\bm{B}}

    %Magnetic field vector -> H in bold
    \providecommand{\bH}{\bm{H}}

    %Current density vector -> J in bold
    \providecommand{\bJ}{\bm{J}}

    %Sheet current vector -> j in bold
    \providecommand{\bj}{\bm{j}}

    %Polarization density vector -> P in bold
    \providecommand{\bP}{\bm{P}}

    %Magnetization density vector -> M in bold
    \providecommand{\bM}{\bm{M}}

%%%%%%%%ELSE STARTS HERE%%%%%%%%%%%%%%%%%%5
\else
    %Electric field vector
    \providecommand{\bE}{\vb{E}}

    %Electric displacement vector
    \providecommand{\bD}{\vb{D}}


    %Potential vector
    \providecommand{\bA}{\vb{A}}
    \providecommand{\hA}{\hat{\vb{A}}}
    \providecommand{\nhA}{\vb{A}} %In case I want to remove the hat and recover it back again

    %Magnetic induction field vector
    \providecommand{\bB}{\vb{B}}

    %Magnetic field vector -> H in bold
    \providecommand{\bH}{\vb{H}}

    %Current density vector -> J in bold
    \providecommand{\bJ}{\vb{J}}

    %Sheet current vector -> j in bold
    \providecommand{\bj}{\vb{j}}

    %Polarization density vector -> P in bold
    \providecommand{\bP}{\vb{P}}

    %Magnetization density vector -> M in bold
    \providecommand{\bM}{\vb{M}}
\fi


%Wave vector
\newcommand{\bk}{\vb{k}}
\newcommand{\bq}{\vb{q}}
\newcommand{\bp}{\vb{p}}

%Corners of graphene Brillouin zone
\newcommand{\bK}{\vb{K}}

%x, y and z versors
\newcommand{\bex}{\vb{e}_x}
\newcommand{\bey}{\vb{e}_y}
\newcommand{\bez}{\vb{e}_z}

\newcommand{\bux}{\vb{u}_x}
\newcommand{\buy}{\vb{u}_y}
\newcommand{\buz}{\vb{u}_z}

%Unit vector for any polarization
\newcommand{\bhe}{\hat{\vb{e}}}

%small electric and magnetic fields
\newcommand{\be}{\vb{e}}
%\newcommand{\bb}{\vb{b}} already defined


%Imaginary unit
\newcommand{\ii}{\mathrm{i}}

%Euler number
\newcommand{\ee}{\mathrm{e}}

%Exponential
\newcommand{\ex}[1]{\mathrm{e}^{#1}}

%Imaginary exponential
\newcommand{\ei}[2][]{\ee^{#1 \ii #2}} %  First argument is for the sign, second argument is the argument of the exponential. If no sign is provided, it defaults to a plus sign. If you want to provide the sign, use \ei[-]{\omega t}. Do not use any other thing as argument


%TM and TE modes in math mode
\newcommand{\TE}{\mathrm{TE}}
\newcommand{\TM}{\mathrm{TM}}

%creation operators
\newcommand{\adagger}{a^{\dagger}}
\newcommand{\bdagger}{b^{\dagger}}
\newcommand{\cdagger}{c^{\dagger}}

% hbar omega
\newcommand{\hw}{\hbar \omega}

% d for single variable variable integrals
%\newcommand{\d}[1][]{d}
%\newcommand{\dife}[1][]{\dd #1} % \d already exists

% differential for time
% t
\newcommand{\dt}{\dd{t}}

% differential for real and reciprocal space integrals
% r vector
\newcommand{\dr}[1][]{\dd[#1]{\br}}

% x&y vector
\newcommand{\dx}[1][]{\dd[#1]{\bx}}
\newcommand{\dy}[1][]{\dd[#1]{\by}}

% k,p,q vector
\newcommand{\dk}[1][]{\dd[#1]{\bk}}
\newcommand{\dpp}[1][]{\dd[#1]{\bp}}
\newcommand{\dq}[1][]{\dd[#1]{\bq}}

%Differential operators
%\newcommand{\grad}{\mathnormal{\nabla}}\mathnormal{\nabla} % physics package already has \grad as bold nabla
%\newcommand{\pnabla}{\mathnormal{\nabla}} % plain nabla
\newcommand{\lap}{\laplacian} % adds another shortcut for the curl operator that exists already in the physics package. In portuguese curl translates to `rotacional'. Using rot instead of the nabla symbol also appears in some literatures. Also in german the word for curl begins with `rot'.
%\renewcommand{\grad}{\bm{\nabla}} % redefines gradient command to change symbol to italic. Comment if unwished.
\newcommand{\rot}{\bm{\nabla} \times} % redefines curl command to change symbol to italic. Comment if unwished.
%\renewcommand{\div}{\bm{\nabla} \cdot} % redefines divergence command to change symbol to italic. Comment if unwished.

% Partial derivatives of first order

\newcommand{\pdt}[1][]{\pdv{#1}{t}} % w.r.t. time
\newcommand{\pdti}[1][]{\pdv*{t}} % w.r.t. time inline

\newcommand{\pdx}[1][]{\pdv{#1}{x}} % w.r.t. x, when calling the command, provide the function to be derivated with []
\newcommand{\pdy}[1][]{\pdv{#1}{y}} % w.r.t. y, when calling the command, provide the function to be derivated with []
\newcommand{\pdz}[1][]{\pdv{#1}{z}} % w.r.t. z, when calling the command, provide the function to be derivated with []

\newcommand{\delt}{\partial_t} % w.r.t. t in subscript
\newcommand{\delx}{\partial_x} % w.r.t. x in subscript
\newcommand{\dely}{\partial_y} % w.r.t. y in subscript
\newcommand{\delz}{\partial_z} % w.r.t. z in subscript

% Partial derivatives of second order
\newcommand{\pdd}[2][]{\pdv[2]{#1}{#2}} % w.r.t. whatever (derivative variable mandatory, what to be derivated optional)

\newcommand{\pddt}[1][]{\pdd[#1]{t}} % w.r.t. time

\newcommand{\pddx}[1][]{\pdd[#1]{x}} % w.r.t. x
\newcommand{\pddy}[1][]{\pdd[#1]{y}} % w.r.t. y
\newcommand{\pddz}[1][]{\pdd[#1]{z}} % w.r.t. z

%\newcommand{\pddix}[1][]{\pddi[#1]{x}} % w.r.t. x inline
%\newcommand{\pddiy}[1][]{\pddi[#1]{y}} % w.r.t. y inline
%\newcommand{\pddiz}[1][]{\pddi[#1]{z}} % w.r.t. z inline

\newcommand{\ddelt}{\partial^2_t} % w.r.t. t in subscript
\newcommand{\ddelx}{\partial^2_x} % w.r.t. x in subscript
\newcommand{\ddely}{\partial^2_y} % w.r.t. y in subscript
\newcommand{\ddelz}{\partial^2_z} % w.r.t. z in subscript

% Total derivatives
\newcommand{\tdt}[1][]{\dv{#1}{t}} % w.r.t. time

% Tensorial symbols

% Kronecker delta
\newcommand{\Kd}[1]{\delta_{#1}} % Kronecker delta without comma
\newcommand{\Kdc}[2]{\delta_{#1,#2}} % Kronecker delta with comma


% Physical quantities

% plasma frequency
\renewcommand{\wp}{\omega_{\text{p}}} % We have to redefine the command, as by default in the amsmath package it stands for the Weierstrass function (the curly P)

% Fermi energy
\newcommand{\EF}{E_F} % We have to redefine the command, as by default in the amsmath package it stands for the Weierstrass function (the curly P)